% Options for packages loaded elsewhere
\PassOptionsToPackage{unicode}{hyperref}
\PassOptionsToPackage{hyphens}{url}
%
\documentclass[
  english,
  man]{apa6}
\usepackage{amsmath,amssymb}
\usepackage{lmodern}
\usepackage{ifxetex,ifluatex}
\ifnum 0\ifxetex 1\fi\ifluatex 1\fi=0 % if pdftex
  \usepackage[T1]{fontenc}
  \usepackage[utf8]{inputenc}
  \usepackage{textcomp} % provide euro and other symbols
\else % if luatex or xetex
  \usepackage{unicode-math}
  \defaultfontfeatures{Scale=MatchLowercase}
  \defaultfontfeatures[\rmfamily]{Ligatures=TeX,Scale=1}
\fi
% Use upquote if available, for straight quotes in verbatim environments
\IfFileExists{upquote.sty}{\usepackage{upquote}}{}
\IfFileExists{microtype.sty}{% use microtype if available
  \usepackage[]{microtype}
  \UseMicrotypeSet[protrusion]{basicmath} % disable protrusion for tt fonts
}{}
\makeatletter
\@ifundefined{KOMAClassName}{% if non-KOMA class
  \IfFileExists{parskip.sty}{%
    \usepackage{parskip}
  }{% else
    \setlength{\parindent}{0pt}
    \setlength{\parskip}{6pt plus 2pt minus 1pt}}
}{% if KOMA class
  \KOMAoptions{parskip=half}}
\makeatother
\usepackage{xcolor}
\IfFileExists{xurl.sty}{\usepackage{xurl}}{} % add URL line breaks if available
\IfFileExists{bookmark.sty}{\usepackage{bookmark}}{\usepackage{hyperref}}
\hypersetup{
  pdftitle={fractalRegression: An R package for multiscale regression and fractal analyses},
  pdfauthor={Aaron D. Likens1 \& Travis J. Wiltshire2},
  pdflang={en-EN},
  pdfkeywords={long range correlation, fractal, multiscale, dynamics},
  hidelinks,
  pdfcreator={LaTeX via pandoc}}
\urlstyle{same} % disable monospaced font for URLs
\usepackage{longtable,booktabs,array}
\usepackage{calc} % for calculating minipage widths
% Correct order of tables after \paragraph or \subparagraph
\usepackage{etoolbox}
\makeatletter
\patchcmd\longtable{\par}{\if@noskipsec\mbox{}\fi\par}{}{}
\makeatother
% Allow footnotes in longtable head/foot
\IfFileExists{footnotehyper.sty}{\usepackage{footnotehyper}}{\usepackage{footnote}}
\makesavenoteenv{longtable}
\usepackage{graphicx}
\makeatletter
\def\maxwidth{\ifdim\Gin@nat@width>\linewidth\linewidth\else\Gin@nat@width\fi}
\def\maxheight{\ifdim\Gin@nat@height>\textheight\textheight\else\Gin@nat@height\fi}
\makeatother
% Scale images if necessary, so that they will not overflow the page
% margins by default, and it is still possible to overwrite the defaults
% using explicit options in \includegraphics[width, height, ...]{}
\setkeys{Gin}{width=\maxwidth,height=\maxheight,keepaspectratio}
% Set default figure placement to htbp
\makeatletter
\def\fps@figure{htbp}
\makeatother
\setlength{\emergencystretch}{3em} % prevent overfull lines
\providecommand{\tightlist}{%
  \setlength{\itemsep}{0pt}\setlength{\parskip}{0pt}}
\setcounter{secnumdepth}{-\maxdimen} % remove section numbering
% Make \paragraph and \subparagraph free-standing
\ifx\paragraph\undefined\else
  \let\oldparagraph\paragraph
  \renewcommand{\paragraph}[1]{\oldparagraph{#1}\mbox{}}
\fi
\ifx\subparagraph\undefined\else
  \let\oldsubparagraph\subparagraph
  \renewcommand{\subparagraph}[1]{\oldsubparagraph{#1}\mbox{}}
\fi
% Manuscript styling
\usepackage{upgreek}
\captionsetup{font=singlespacing,justification=justified}

% Table formatting
\usepackage{longtable}
\usepackage{lscape}
% \usepackage[counterclockwise]{rotating}   % Landscape page setup for large tables
\usepackage{multirow}		% Table styling
\usepackage{tabularx}		% Control Column width
\usepackage[flushleft]{threeparttable}	% Allows for three part tables with a specified notes section
\usepackage{threeparttablex}            % Lets threeparttable work with longtable

% Create new environments so endfloat can handle them
% \newenvironment{ltable}
%   {\begin{landscape}\centering\begin{threeparttable}}
%   {\end{threeparttable}\end{landscape}}
\newenvironment{lltable}{\begin{landscape}\centering\begin{ThreePartTable}}{\end{ThreePartTable}\end{landscape}}

% Enables adjusting longtable caption width to table width
% Solution found at http://golatex.de/longtable-mit-caption-so-breit-wie-die-tabelle-t15767.html
\makeatletter
\newcommand\LastLTentrywidth{1em}
\newlength\longtablewidth
\setlength{\longtablewidth}{1in}
\newcommand{\getlongtablewidth}{\begingroup \ifcsname LT@\roman{LT@tables}\endcsname \global\longtablewidth=0pt \renewcommand{\LT@entry}[2]{\global\advance\longtablewidth by ##2\relax\gdef\LastLTentrywidth{##2}}\@nameuse{LT@\roman{LT@tables}} \fi \endgroup}

% \setlength{\parindent}{0.5in}
% \setlength{\parskip}{0pt plus 0pt minus 0pt}

% \usepackage{etoolbox}
\makeatletter
\patchcmd{\HyOrg@maketitle}
  {\section{\normalfont\normalsize\abstractname}}
  {\section*{\normalfont\normalsize\abstractname}}
  {}{\typeout{Failed to patch abstract.}}
\patchcmd{\HyOrg@maketitle}
  {\section{\protect\normalfont{\@title}}}
  {\section*{\protect\normalfont{\@title}}}
  {}{\typeout{Failed to patch title.}}
\makeatother
\shorttitle{FRACTAL REGRESSION}
\keywords{long range correlation, fractal, multiscale, dynamics\newline\indent Word count: X}
\DeclareDelayedFloatFlavor{ThreePartTable}{table}
\DeclareDelayedFloatFlavor{lltable}{table}
\DeclareDelayedFloatFlavor*{longtable}{table}
\makeatletter
\renewcommand{\efloat@iwrite}[1]{\immediate\expandafter\protected@write\csname efloat@post#1\endcsname{}}
\makeatother
\usepackage{lineno}

\linenumbers
\usepackage{csquotes}
\ifxetex
  % Load polyglossia as late as possible: uses bidi with RTL langages (e.g. Hebrew, Arabic)
  \usepackage{polyglossia}
  \setmainlanguage[]{english}
\else
  \usepackage[main=english]{babel}
% get rid of language-specific shorthands (see #6817):
\let\LanguageShortHands\languageshorthands
\def\languageshorthands#1{}
\fi
\ifluatex
  \usepackage{selnolig}  % disable illegal ligatures
\fi
\newlength{\cslhangindent}
\setlength{\cslhangindent}{1.5em}
\newlength{\csllabelwidth}
\setlength{\csllabelwidth}{3em}
\newenvironment{CSLReferences}[2] % #1 hanging-ident, #2 entry spacing
 {% don't indent paragraphs
  \setlength{\parindent}{0pt}
  % turn on hanging indent if param 1 is 1
  \ifodd #1 \everypar{\setlength{\hangindent}{\cslhangindent}}\ignorespaces\fi
  % set entry spacing
  \ifnum #2 > 0
  \setlength{\parskip}{#2\baselineskip}
  \fi
 }%
 {}
\usepackage{calc}
\newcommand{\CSLBlock}[1]{#1\hfill\break}
\newcommand{\CSLLeftMargin}[1]{\parbox[t]{\csllabelwidth}{#1}}
\newcommand{\CSLRightInline}[1]{\parbox[t]{\linewidth - \csllabelwidth}{#1}\break}
\newcommand{\CSLIndent}[1]{\hspace{\cslhangindent}#1}

\title{fractalRegression: An R package for multiscale regression and fractal analyses}
\author{Aaron D. Likens\textsuperscript{1} \& Travis J. Wiltshire\textsuperscript{2}}
\date{}


\authornote{

Add complete departmental affiliations for each author here. Each new line herein must be indented, like this line.
Enter author note here.

The authors made the following contributions. Aaron D. Likens: Conceptualization, Software, Writing - Original Draft Preparation, Writing - Review \& Editing; Travis J. Wiltshire: Writing - Original Draft Preparation, Writing - Review \& Editing, Software.

Correspondence concerning this article should be addressed to Aaron D. Likens, Postal address. E-mail: \href{mailto:alikens@unomaha.edu}{\nolinkurl{alikens@unomaha.edu}}

}

\affiliation{\vspace{0.5cm}\textsuperscript{1} Department of Biomechanics, University of Nebraska at Omaha\\\textsuperscript{2} Department of Cognitive Science \& Artificial Intelligence, Tilburg University}

\abstract{
Time series data from scientific fields as diverse as astrophysics, economics, human movement science, and neuroscience all exhibit fractal properties. That is, these time series often exhibit self-similarity and long-range correlations. This \texttt{fractalRegression} package implements a number of univariate and bivariate time series tools appropriate for analyzing noisy data exhibiting these properties. These methods, especially the bivariate tools (Kristoufek, 2015; Likens, Amazeen, West, \& Gibbons, 2019) have yet to be implemented in an open source and complete package for the R Statistical Software environment. As both practitioners and developers of these methods, we expect these tools will be of interest to a wide audience of scientists who use R, especially those from fields such as the human movement, cognitive, and other behavioral sciences. The algorithms have been developed in C++ using the popular Rcpp (Eddelbuettel \& Francois, 2011) and RcppArmadillo (Eddelbuettel \& Sanderson, 2014) packages. The result is a collection of efficient functions that perform well even on long time series (e.g., \(\geq\) 10,000 data points). In this work, we motivate introduce the package, each of the functions, and give examples of their use as well as issues to consider to correctly use these methods.
}



\begin{document}
\maketitle

\hypertarget{introduction}{%
\section{Introduction}\label{introduction}}

Fractal analysis, in its many forms, has become an important framework
in virtually every area of science, often serving as an indicator of
system health (Goldberger et al., 2002), adaptability
(Bak, Tang, \& Wiesenfeld, 1987), control
(Likens, Fine, Amazeen, \& Amazeen, 2015), cognitive function
(Euler, Wiltshire, Niermeyer, \& Butner, 2016), and multi-scale interactions
(Kelty-Stephen, 2017). In particular, various
methods related to Detrended Fluctuation Analysis (DFA)
(Peng et al., 1994) have rose to prominence due to their
ease of understanding and broad applicability to stationary and
nonstationary time series, alike. The basic DFA algorithm has been
implemented in numerous packages and software programs. However,
advanced methods such as Multifractal Detrended Fluctuation Analysis
(MFDFA) (Jan W. Kantelhardt et al., 2002), Detrended
Cross Correlation (DCCA) (Podobnik \& Stanley, 2008; Zebende, 2011), and, in particular,
fractal regression techniques such as Multiscale Regression Analysis
(MRA) (Kristoufek, 2015; Likens, Amazeen, West, \& Gibbons, 2019) have not yet been
implemented in a comprehensive CRAN Package for the R Statistical
Software Environment. Thus, there is a clear need for a package that
incorporates this functionality in order to advance theoretical research
focused on understanding the time varying properties of natural
phenomena and applied research that uses those insights in important
areas such as healthcare (Cavanaugh, Kelty-Stephen, \& Stergiou, 2017) and education (Snow, Likens, Allen, \& McNamara, 2016).

\hypertarget{package-overview}{%
\section{Package Overview}\label{package-overview}}

Some foundational efforts in fractal analyses, which partially overlap
with the functionality of this package, have been implemented elsewhere.
For example, a number of univariate fractal and multifractal analyses
have been implemented in the `fracLab' library for MATLAB (Legrand \& Véhel, 2003)
and other toolboxes that are mainly targeted at multifractal analysis
(E. A. F. E. A. F. I. Ihlen, 2012; E. A. F. Ihlen \& Vereijken, 2010). In terms of
open access packages, there are other packages that implement some, but
not all of the same functions such as the \texttt{fathon} package
(Bianchi, 2020) that has been implemented in Python as well as the R
packages: \texttt{fractal} {[}@{]}, \texttt{nonlinearTseries}
(Garcia, 2020), and \texttt{MFDFA}
(Laib, Golay, Telesca, \& Kanevski, 2018). However, none of the above packages
incorporate univariate monofractal and multifractal DFA with bivariate
DCCA and MRA nor do they run on a C++ architecture. Our
\texttt{fractalRegression} package is unique in this combination of analyses
and efficiency. For instance, we are not aware of any other packages
that feature MRA and Multiscale Lagged Regression (MLRA).

\hypertarget{methodological-details-and-examples}{%
\section{Methodological Details and Examples}\label{methodological-details-and-examples}}

In order to demonstrate the methods within the `fractalRegression'
package, we group this into univariate (DFA, MFDFA) and bivariate
methods (DCCA, MRA, MRLA). For each method, we 1) highlight the key
question(s) that can be answered with that method, 2) briefly describe
the algorithm with sources for additional details, 3) describe some key
consideration for appropriately applying the algorithm, and demonstrate
the use of the functions on a 4) simulated and 5) empirical application
of the function.

Table 1. Overview of package functions, objectives, and output.

\begin{longtable}[]{@{}lll@{}}
\toprule
Function & Objective & Output \\
\midrule
\endhead
& & \\
& & \\
& & \\
& & \\
& & \\
& & \\
\bottomrule
\end{longtable}

\hypertarget{univariate-methods}{%
\subsection{Univariate Methods}\label{univariate-methods}}

\hypertarget{detrended-fluctuation-analysis}{%
\subsubsection{Detrended Fluctuation Analysis}\label{detrended-fluctuation-analysis}}

The key questions that can be answered by Detrended Fluctuation Analysis
(DFA) (Peng et al., 1994) is INSERT KEY QUESTION HERE. DFA
has been described extensively elsewhere
(Jan W. Kantelhardt, Koscielny-Bunde, Rego, Havlin, \& Bunde, 2001) and visualized nicely
(Kelty-Stephen, Stirling, \& Lipsitz, 2016); so we provide a
brief summary here. DFA entails splitting a time series into several
small bins (e.g., 16). In each bin, the least squares regression is fit
and subtracted within each window. Residuals are squared and averaged
within each window. Then, the square root is taken of the average
squared residual across all windows of a given size. This process
repeats for larger window sizes, growing by, say a power of 2, up to
N/4, where N is the length of the series. In a final step, the logarithm
of those scaled root mean squared residuals (i.e., fluctuations) is
regressed on the logarithm of window sizes. The slope of this line is
termed \(\alpha\) and it provides a measure of the long range correlation.
Conventional interpretation of \(\alpha\) is:

\begin{itemize}
\tightlist
\item
  \(\alpha < 0.5 =\) anti-correlated
\item
  \(\alpha ~= 0.5 =\) uncorrelated, white noise
\item
  \(\alpha > 0.5 =\) temporally correlated
\item
  \(\alpha ~= 1 =\) 1/f-noise, pink noise
\item
  \(\alpha > 1 =\) non-stationary and unbounded
\item
  \(\alpha ~= 1.5 =\) fractional brownian motion
\end{itemize}

\hypertarget{dfa-examples}{%
\paragraph{DFA Examples}\label{dfa-examples}}

INCLUDE EXAMPLES HERE

\hypertarget{dfa-considerations}{%
\paragraph{DFA Considerations}\label{dfa-considerations}}

We recommend a few points of consideration here in using this function.
One is to be sure to verify there are not cross-over points in the
logScale-logFluctuation plots (Peng et al., 1995; Perakakis et al .,
2009). Cross-over points (or a visible change in the slope as a function
of of scale) indicate that a mono-fractal characterization does not
sufficiently characterize the data. If cross-over points are evident, we
recommend proceeding to using the `mfdfa()' to estimate the
multi-fractal fluctuation dynamics across scales.

While it is common to use only linear detrending with DFA, it is
important to inspect the trends in the data to determine if it would be
more appropriate to use a higher order polynomial for detrending, and/or
compare the DFA output for different polynomial orders (see Kantelhardt
et al., 2001).

General recommendations for choosing the min and max scale are an sc\_min
= 10 and sc\_max = (N/4), where N is the number of observations. See Eke
et al.~(2002) and Gulich and Zunino (2014) for additional
considerations. - Key Question and method description. - Simulated data:
fGN - Empirical data: EPICLE Movement Data? - MFDFA - Key Question(s) -
Simulated data: mfbrownian motion from Ihlen matlab (Aaron might have R
port) see mbm\_mgn for R from aaron - Empirical data: EPICLE Movement
Data?

\hypertarget{multifractal-detrended-fluctuation-analysis}{%
\subsubsection{Multifractal Detrended Fluctuation Analysis}\label{multifractal-detrended-fluctuation-analysis}}

Multifractal Detrended Fluctuation Analysis (MFDFA; Kantelhardt et al.,
2002) is an extension of DFA by generalizing the fluctuation function to
a range of exponents of the qth order.

\hypertarget{bivariate-methods}{%
\subsection{Bivariate Methods}\label{bivariate-methods}}

\begin{itemize}
\item
  DCCA

  \begin{itemize}
  \tightlist
  \item
    Key Question: What the timescale(s) of coordination?
  \item
    Simulated data: MC-ARFIMA
  \item
    Empirical data: EPICLE Movement Data?
  \end{itemize}
\item
  MRA

  \begin{itemize}
  \tightlist
  \item
    Key Question
  \item
    Simulated data:
  \item
    Empirical data: FNIRS from Aaron?
  \end{itemize}
\item
  MLRA

  \begin{itemize}
  \tightlist
  \item
    Key Question
  \item
    Simulated data: Equation from Aaron from grant on MLRA
  \item
    Empirical data: FNIRS from Aaron?
  \end{itemize}
\end{itemize}

\hypertarget{surrogate-methods-and-full-data-analysis}{%
\subsection{Surrogate Methods (and `full' data analysis)}\label{surrogate-methods-and-full-data-analysis}}

Methods are ranked in terms of increasing levels of rigor.

\begin{itemize}
\tightlist
\item
  Randomization - Estimates should be different.
\item
  IAAFT - Estimates should be different.
\item
  Model based surrogate (Simulated exponents) - See Likens 2019 paper
  with model of postural sway/control, taking an educated guess about
  the data generating process underlying the time series. Estimates
  should not be different. See Roume et al 2018 windowed detrended CCA
\item
  Can we incorporate lags into MC-ARFIMA?
\end{itemize}

\hypertarget{general-discussion}{%
\section{General Discussion}\label{general-discussion}}

\begin{itemize}
\tightlist
\item
  General value of methods and the types of questions
\item
  Practical consideration of univariate methods
\item
  Practical consideration of bivariate methods
\item
  Unique contribution of the methods
\end{itemize}

\hypertarget{acknowledgements}{%
\section{Acknowledgements}\label{acknowledgements}}

Author AL receives support from a National Institutes of Health Center
grant (P20GM109090).

\newpage

\hypertarget{references}{%
\section{References}\label{references}}

\begingroup
\setlength{\parindent}{-0.5in}
\setlength{\leftskip}{0.5in}

\hypertarget{refs}{}
\begin{CSLReferences}{1}{0}
\leavevmode\hypertarget{ref-bakSelforganizedCriticalityExplanation1987}{}%
Bak, P., Tang, C., \& Wiesenfeld, K. (1987). Self-organized criticality: {An} explanation of the 1/f noise. \emph{Physical Review Letters}, \emph{59}(4), 381--384. \url{https://doi.org/10.1103/PhysRevLett.59.381}

\leavevmode\hypertarget{ref-bianchi2020}{}%
Bianchi, S. (2020). Fathon: A python package for a fast computation of detrendend fluctuation analysis and related algorithms. \emph{Journal of Open Source Software}, \emph{5}(45), 1828.

\leavevmode\hypertarget{ref-cavanaugh2017}{}%
Cavanaugh, J. T., Kelty-Stephen, D. G., \& Stergiou, N. (2017). Multifractality, Interactivity, and the Adaptive Capacity of the Human Movement System: A Perspective for Advancing the Conceptual Basis of Neurologic Physical Therapy. Retrieved from \url{https://www.ingentaconnect.com/content/wk/npt/2017/00000041/00000004/art00007}

\leavevmode\hypertarget{ref-eddelbuettelRcppSeamlessIntegration2011}{}%
Eddelbuettel, D., \& Francois, R. (2011). Rcpp: {Seamless} {R} and {C}++ {Integration}. \emph{Journal of Statistical Software}, \emph{40}(1), 1--18. \url{https://doi.org/10.18637/jss.v040.i08}

\leavevmode\hypertarget{ref-eddelbuettelRcppArmadilloAcceleratingHighperformance2014}{}%
Eddelbuettel, D., \& Sanderson, C. (2014). {RcppArmadillo}: {Accelerating} {R} with high-performance {C}++~linear algebra. \emph{Computational Statistics \& Data Analysis}, \emph{71}, 1054--1063. \url{https://doi.org/10.1016/j.csda.2013.02.005}

\leavevmode\hypertarget{ref-eulerWorkingMemoryPerformance2016}{}%
Euler, M. J., Wiltshire, T. J., Niermeyer, M. A., \& Butner, J. E. (2016). Working memory performance inversely predicts spontaneous delta and theta-band scaling relations. \emph{Brain Research}, \emph{1637}, 22--33. \url{https://doi.org/10.1016/j.brainres.2016.02.008}

\leavevmode\hypertarget{ref-garciaNonlinearTseriesNonlinearTime2020}{}%
Garcia, C. A. (2020). {nonlinearTseries}: {Nonlinear} {Time} {Series} {Analysis}. Retrieved from \url{https://CRAN.R-project.org/package=nonlinearTseries}

\leavevmode\hypertarget{ref-goldbergerFractalDynamicsPhysiology2002}{}%
Goldberger, A. L., Amaral, L. A. N., Hausdorff, J. M., Ivanov, P. C., Peng, C.-K., \& Stanley, H. E. (2002). Fractal dynamics in physiology: {Alterations} with disease and aging. \emph{Proceedings of the National Academy of Sciences}, \emph{99}(suppl 1), 2466--2472. \url{https://doi.org/10.1073/pnas.012579499}

\leavevmode\hypertarget{ref-ihlenIntroductionMultifractalDetrended2012}{}%
Ihlen, E. A. F. E. A. F. I. (2012). Introduction to {Multifractal} {Detrended} {Fluctuation} {Analysis} in {Matlab}. \emph{Frontiers in Physiology}, \emph{3}. \url{https://doi.org/10.3389/fphys.2012.00141}

\leavevmode\hypertarget{ref-ihlen2010}{}%
Ihlen, E. A. F., \& Vereijken, B. (2010). Interaction-dominant dynamics in human cognition: Beyond 1/{{}}? fluctuation. \emph{Journal of Experimental Psychology: General}, \emph{139}(3), 436--463. \url{https://doi.org/10.1037/a0019098}

\leavevmode\hypertarget{ref-kantelhardtDetectingLongrangeCorrelations2001}{}%
Kantelhardt, Jan W., Koscielny-Bunde, E., Rego, H. H. A., Havlin, S., \& Bunde, A. (2001). Detecting long-range correlations with detrended fluctuation analysis. \emph{Physica A: Statistical Mechanics and Its Applications}, \emph{295}(3), 441--454. \url{https://doi.org/10.1016/S0378-4371(01)00144-3}

\leavevmode\hypertarget{ref-kantelhardtMultifractalDetrendedFluctuation2002}{}%
Kantelhardt, Jan W., Zschiegner, S. A., Koscielny-Bunde, E., Havlin, S., Bunde, A., \& Stanley, H. E. (2002). Multifractal detrended fluctuation analysis of nonstationary time series. \emph{Physica A: Statistical Mechanics and Its Applications}, \emph{316}(1), 87--114. \url{https://doi.org/10.1016/S0378-4371(02)01383-3}

\leavevmode\hypertarget{ref-kelty-stephenThreadingMultifractalSocial2017}{}%
Kelty-Stephen, D. G. (2017). Threading a multifractal social psychology through within-organism coordination to within-group interactions: {A} tale of coordination in three acts. \emph{Chaos, Solitons \& Fractals}, \emph{104}, 363--370. \url{https://doi.org/10.1016/j.chaos.2017.08.037}

\leavevmode\hypertarget{ref-kelty-stephenMultifractalTemporalCorrelations2016}{}%
Kelty-Stephen, D. G., Stirling, L. A., \& Lipsitz, L. A. (2016). Multifractal temporal correlations in circle-tracing behaviors are associated with the executive function of rule-switching assessed by the {Trail} {Making} {Test}. \emph{Psychological Assessment}, \emph{28}(2), 171--180. \url{https://doi.org/10.1037/pas0000177}

\leavevmode\hypertarget{ref-kristoufekDetrendedFluctuationAnalysis2015}{}%
Kristoufek, L. (2015). Detrended fluctuation analysis as a regression framework: {Estimating} dependence at different scales. \emph{Physical Review E}, \emph{91}(2), 022802. \url{https://doi.org/10.1103/PhysRevE.91.022802}

\leavevmode\hypertarget{ref-laibMultifractalAnalysisTime2018}{}%
Laib, M., Golay, J., Telesca, L., \& Kanevski, M. (2018). Multifractal analysis of the time series of daily means of wind speed in complex regions. \emph{Chaos, Solitons \& Fractals}, \emph{109}, 118--127. \url{https://doi.org/10.1016/j.chaos.2018.02.024}

\leavevmode\hypertarget{ref-legrand2003}{}%
Legrand, P., \& Véhel, J. L. (2003). Signal and image processing with FracLab. \emph{Thinking in Patterns: Fractals and Related Phenomena in Nature}, 321322.

\leavevmode\hypertarget{ref-likensStatisticalPropertiesMultiscale2019}{}%
Likens, A. D., Amazeen, P. G., West, S. G., \& Gibbons, C. T. (2019). Statistical properties of {Multiscale} {Regression} {Analysis}: {Simulation} and application to human postural control. \emph{Physica A: Statistical Mechanics and Its Applications}, \emph{532}, 121580. \url{https://doi.org/10.1016/j.physa.2019.121580}

\leavevmode\hypertarget{ref-likensExperimentalControlScaling2015}{}%
Likens, A. D., Fine, J. M., Amazeen, E. L., \& Amazeen, P. G. (2015). Experimental control of scaling behavior: What is not fractal? \emph{Experimental Brain Research}, \emph{233}(10), 2813--2821. \url{https://doi.org/10.1007/s00221-015-4351-4}

\leavevmode\hypertarget{ref-pengMosaicOrganizationDNA1994}{}%
Peng, C.-K., Buldyrev, S. V., Havlin, S., Simons, M., Stanley, H. E., \& Goldberger, A. L. (1994). Mosaic organization of {DNA} nucleotides. \emph{Physical Review E}, \emph{49}(2), 1685--1689. \url{https://doi.org/10.1103/PhysRevE.49.1685}

\leavevmode\hypertarget{ref-podobnikDetrendedCrossCorrelationAnalysis2008}{}%
Podobnik, B., \& Stanley, H. E. (2008). Detrended {Cross}-{Correlation} {Analysis}: {A} {New} {Method} for {Analyzing} {Two} {Nonstationary} {Time} {Series}. \emph{Physical Review Letters}, \emph{100}(8), 084102. \url{https://doi.org/10.1103/PhysRevLett.100.084102}

\leavevmode\hypertarget{ref-snow2016}{}%
Snow, E. L., Likens, A. D., Allen, L. K., \& McNamara, D. S. (2016). Taking Control: Stealth Assessment of Deterministic Behaviors Within a Game-Based System. \emph{International Journal of Artificial Intelligence in Education}, \emph{26}(4), 1011--1032. \url{https://doi.org/10.1007/s40593-015-0085-5}

\leavevmode\hypertarget{ref-zebendeDCCACrosscorrelationCoefficient2011}{}%
Zebende, G. F. (2011). {DCCA} cross-correlation coefficient: {Quantifying} level of cross-correlation. \emph{Physica A: Statistical Mechanics and Its Applications}, \emph{390}(4), 614--618. \url{https://doi.org/10.1016/j.physa.2010.10.022}

\end{CSLReferences}

\endgroup


\end{document}
